\documentclass[a4paper]{article}
%\usepackage[affil-it]{authblk}
\usepackage[backend=bibtex,style=numeric]{biblatex}
\usepackage{amsmath}
\usepackage{geometry}
\usepackage{caption}
\usepackage{amssymb}
\usepackage{ctex}
\geometry{margin=1.5cm, vmargin={0pt,1cm}}
\setlength{\topmargin}{-1cm}
\setlength{\paperheight}{29.7cm}
\setlength{\textheight}{25.3cm}

\begin{document}
% ==================================================
\title{Numerical Analysis Homework 1}

\author{罗开诚 Luo Kaicheng 3220103383
  \thanks{Electronic address: \texttt{3220103383@zju.edu.com}}}
%\affil{(Information and Computational Science 2201), Zhejiang University}

\date{Due time: \today}

\maketitle

\begin{abstract}
    Solutions to various numerical analysis problems.
\end{abstract}

% ============================================

\section*{I}
1. We have \( f(x_0) = 1, f(x_1) = \frac{1}{2} \). \\
Using the interpolation method, we obtained \\
\( p_1(f; x) = f(x_0)*\frac{x-x_1}{x_0-x_1}+f(x_1)*\frac{x-x_0}{x_1-x_0}= -\frac{1}{2}x + \frac{3}{2} \). \\
Therefore, \( f(x) - p_1(f; x) = \frac{1}{x} + \frac{x}{2} - \frac{3}{2} = \frac{f''(\xi(x))}{2}(x - 1)(x - 2) \). \\
\( \xi(x) = (2x)^{\frac{1}{3}} \).
\\
2. After extending the domain of \( \xi \) continuously from \( (x_0, x_1) \) to \( [x_0, x_1] \), \\
we have \( \xi(x_0) = \lim_{x \to x_0} \xi(x) = 2^{\frac{1}{3}}, \xi(x_1) = \lim_{x \to x_1} \xi(x) = 4^{\frac{1}{3}} \). \( \xi'(x) = \frac{2}{3}(2x)^{-\frac{2}{3}} > 0 \) in \( [x_0, x_1] \).\\
 Thus, \( \max \xi(x) = 4^{\frac{1}{3}} \), \( \min \xi(x) = 2^{\frac{1}{3}} \), and \( \max f''(\xi(x)) = \max \frac{2}{\xi^3(x)} = 1 \).
\\
\section*{II}
Let
\[
l_k(x) = \prod_{i=0; i \neq k}^n \left( \frac{x - x_i}{x_k - x_i} \right)^2,
\]
\\we have \( l_k \in P_{2n}^{+} \),  \(l_k(x_k)=1,l_k(x_j)=0 (j \neq k) \).\\
Make \( p(x) = \sum_{i=0}^{k} f_i l_k(x) \).\\
It satisfies \( p(x_i) = f_i, p(x) \in P_{2n}^{+} \).

\section*{III}
1. For all \( t \in \mathbb{R} \), we have \( f[t] = f(t) = e^t \), which meets the condition when \( n = 0 \). If it holds for \( n = k \),
when \( n = k + 1 \),
\[
f[t, t+1, \ldots, t+k+1] = \frac{f[t+1, \ldots, t+k+1] - f[t, t+1, \ldots, t+k]}{t+k+1 - t}.
\]
By the induction hypothesis, we know \( f[t+1, \ldots, t+k+1] = \frac{(e-1)^k}{k!} e^{t+1} \),
\( f[t, \ldots, t+k] = \frac{(e-1)^k}{k!} e^t \),\\
thus \( f[t, t+1, \ldots, t+k+1] = \frac{(e-1)^k (e^{t+1} - e^t)}{k! (k+1)} = \frac{(e-1)^{k+1}}{(k+1)!} e^t \) holds.\\
 By the principle of mathematical induction, the conclusion is valid.\\

2. From (1), we know that\\
\[
\forall t \in \mathbb{R}, \quad f[t, t+1, \ldots, t+n] = \frac{(e-1)^n}{n!} e^t.
\]
That is \( f[0, 1, \ldots, n] = \frac{(e-1)^n}{n!} \). \\
Substituting, we get \( \frac{(e-1)^n}{n!} = \frac{1}{n!} f^{(n)}(\xi) =\frac{e^{\xi}}{n!}\),
\( \xi = n \ln(e-1) \),\\
\( \ln(e-1) > 0.5 \).So $\xi$ locates on the right side of \( \frac{n}{2} \).\\

\section*{IV}
1. Construct the divided difference as follows:\\
x$|$ f\\
0$|$ 5\\
1$|$ 3  -2\\
3$|$ 5  1 1\\
4$|$ 12 7 2 0.25\\
We get the result
\[
p_3(f; x) = 5 - 2x + x(x - 1) + \frac{1}{4}x(x - 1)(x - 3) = \frac{1}{4}x^3 - \frac{9}{4}x + 5.
\]

2. Use the minimum point of the interpolation polynomial to estimate the minimum point of \( f \).
\[
p_3(f; x)' = \frac{3}{4}x^2 - \frac{9}{4}.
\]
It is known that \( p_3(f; x) \) is decreasing on \( [-\sqrt{3}, \sqrt{3}] \) and increasing on \( [\sqrt{3}, \infty) \), the estimated minimum point is \( \sqrt{3} \).

\section*{V}
1.
Similarly, construct the divided difference\\
  x\\
  0$|$ 0\\
  1$|$ 1 1\\
  1$|$ 1 7 6\\
  1$|$ 1 7 21 15\\
  2$|$ 128 127 120 99 42\\
  2$|$ 128 448 321 201 102 30\\

\( f[0, 1, 1, 1, 2, 2] = 30 \).

2.
\[
f^{(5)}(\xi) = 2520 \xi^2,
\]
\[
\frac{f^{(5)}(\xi)}{5!} = 21 \xi^2.
\]
Substituting \( 21 \xi^2 = 30 \), we get \( \xi = \frac{\sqrt{70}}{7} \in (0,2) \).

\section*{VI}
1.
Construct the divided difference:\\
  x\\
  0$|$ 1\\
  1$|$ 2 1\\
  1$|$ 2 -1 -2\\
  3$|$ 0 -1 0   0.666667\\
  3$|$ 0 0  0.5 0.25 -0.138889\\
\( p_4(f; x) = \frac{-5}{36}(x-3)(x-1)^2x + \frac{2}{3}x(x-1)^2 - 2x(x-1) + x + 
1 \).\\
Substituting, we get \( f(2) \approx \frac{11}{18} \).

2.
By Theorem 2.37
For the Hermite interpolation problem, denote \( N = k + m_i \). Denote by \( p(f; x) \) the unique element of \( P_n \) that satisfies the conditions. Suppose \( f^{(N+1)}(r) \) exists in \( (a, b) \). Then there exists some \( \xi \in (a, b) \) such that
\[
f(x) - p_N(f; x) = \frac{f^{(N+1)}(\xi)}{(N+1)!} \prod_{i=0}^k (x - x_i)^{m_i + 1}.
\]
In this problem, \( N \) takes 4, \( f^{(5)}(r) \) exists in \( (0, 3) \)
\[
|f(x) - p_4(f; x)| = \left| \frac{f^{(5)}(\xi)}{5!} x (x - 1)^2 (x-3)^2 \right| \leq \frac{M}{5!} x (x - 1)^2 (x-3)^2.
\]
Substitute x=2. We get 
\[
|f(x) - p_4(f; x)|  \leq \frac{M}{5!} 2 (2 - 1)^2 (2-3)^2=\frac{M}{60}.
\]
The maximum possible error is \( |f(x) - p_4(f; x)| \leq M \frac{M}{60} \). Taking \( x = 2 \) and the corresponding \( \xi \) in Theorem 2.37 such that \( f^{(5)}(\xi) = M \) gives the maximum possible error.

\section*{VII}
1. Use mathematical induction for the formula 1,
For any \( x \), if \( k = 1 \):

\(\Delta f(x) = f(x+h) - f(x) = h \frac{f(x+h) - f(x)}{h} = 1! h f[x_0, x_1] \) holds,
Assume the formula holds for any \( x \), \( k = n \), \( k = n + 1 \) when:
\[
\Delta^{n+1} f(x) = \Delta \Delta^n f(x) = \Delta^n f(x+h) - \Delta^n f(x)
\]
By the induction hypothesis, taking \( k = n \), we have
\[
\Delta^n f(x) = n! h^n f[x_0, x_1, \ldots, x_n]
\]
Since the conclusion holds for any \( x \) when \( k = n \), replacing \( x \) with \( x + h \), we get
\[
\Delta^n f(x+h) = n! h^n f[x_1, x_2, \ldots, x_{n+1}]
\]
Substituting

$\Delta^{n+1} f(x) = \Delta \Delta^n f(x) = \Delta^n f(x+h) - \Delta^n f(x) = n! h^n (f[x_1, x_2, \ldots, x_{n+1}] - f[x_0, x_1, \ldots, x_n]) \\ 
=(n+1)! h^{n+1} \frac{f[x_1, x_2, \ldots, x_{n+1}] - f[x_0, x_1, \ldots, x_n]}{(n+1)h}$
\(= (n+1)! h^{n+1} f[x_0, x_1, \ldots, x_{n+1}] \) holds.\\
By mathematical induction, the conclusion is valid.

2. Similarly, prove the second formula.
For any \( x \), if \( k = 1 \):

\(\nabla f(x) = f(x) - f(x-h) = h \frac{f(x) - f(x-h)}{h} = 1! h f[x_0, x_{-1}] \) holds,
Assume the formula holds for any \( x \), \( k = n \), \( k = n + 1 \) when:
\[
\nabla^{n+1} f(x) = \nabla \nabla^n f(x) = \nabla^n f(x) - \nabla^n f(x-h)
\]
By the induction hypothesis, taking \( k = n \), we have
\[
\nabla^n f(x) = n! h^n f[x_0, x_{-1}, \ldots, x_{-n}]
\]
Since the conclusion holds for any \( x \) when \( k = n \), replacing \( x \) with \( x - h \), we get
\[
\nabla^n f(x-h) = n! h^n f[x_{-1}, x_{-2}, \ldots, x_{-(n+1)}]
\]
Substituting

$\nabla^{n+1} f(x) = \nabla \nabla^n f(x) = \nabla^n f(x) - \nabla^n f(x-h) = n! h^n (f[x_0, x_{-1}, \ldots, x_{-n}] - f[x_{-1}, x_{-2}, \ldots, x_{-(n+1)}]) \\
= (n+1)! h^{n+1} \frac{f[x_0, x_{-1}, \ldots, x_{-n}] - f[x_{-1}, x_{-2}, \ldots, x_{-(n+1)}]}{(n+1)h}$
\(= (n+1)! h^{n+1} f[x_0, x_{-1}, \ldots, x_{-{n+1}}] \) holds.\\
By mathematical induction, the conclusion is valid.

\section*{VIII}
Using mathematical induction, when \( n = 0 \):
\[
\frac{\partial}{\partial x_0} f[x_0] = \lim_{\epsilon \to 0} \frac{f[x_0 + \epsilon] - f[x_0]}{\epsilon} = \lim_{\epsilon \to 0} \frac{f(x_0 + \epsilon) - f(x_0)}{\epsilon} = f'(x_0) = f[x_0, x_0]
\]

\(\frac{\partial}{\partial x_0} f[x_0, x_1] = \frac{\partial}{\partial x_0} \frac{f[x_0] - f[x_1]}{x_0 - x_1}\)
\(= \frac{-(f[x_0] - f[x_1])}{(x_0 - x_1)^2} + \frac{\frac{\partial}{\partial x_0} f[x_0]}{x_0 - x_1} = \frac{-f[x_0, x_1] + f[x_0, x_0]}{x_0 - x_1} = f[x_0, x_0, x_1] \) holds.\\
Assume that \( \frac{\partial}{\partial x_0} f[x_0, x_1, \ldots, x_n] = f[x_0, x_0, \ldots, x_n] \) holds for \( n = k \).\\
When \( n = k + 1 \):\\

\(\frac{\partial}{\partial x_0} f[x_0, x_1, \ldots, x_{k+1}] = \frac{\partial}{\partial x_0} \frac{f[x_0, x_1, \ldots, x_k] - f[x_1, x_2, \ldots, x_k]}{x_0 - x_{k+1}}\)
\(= \frac{-(f[x_0, x_1, \ldots, x_k] - f[x_1, x_2, \ldots, x_{k+1}])}{(x_0 - x_{k+1})^2} + \frac{\frac{\partial}{\partial x_0} f[x_0, x_1, \ldots, x_k]}{x_0 - x_{k+1}} = \\
\frac{-f[x_0, x_1, \ldots, x_{k+1}] + f[x_0, x_0, \ldots, x_k]}{x_0 - x_{k+1}} \) (by the induction hypothesis) \( = f[x_0, x_0, x_1, \ldots, x_{k+1}] \) holds.\\
And \( x_k \neq x_0 \), otherwise it should be required that \( f \) is higher-order differentiable at \( x_0 \).\\
By mathematical induction, the conclusion is valid.

If considering the partial derivative with respect to other variables, if \( f \) is differentiable at \( x_i \).\\
\[
\frac{\partial}{\partial x_i} f[x_0, x_1, \ldots, x_i, \ldots, x_n]
\]
\[
= \frac{\partial}{\partial x_i} f[x_i, x_1, \ldots, x_0, \ldots, x_n]=f[x_i, x_i, x_1, \ldots, x_n]
\]
can be proved in the same way.\\
By the uniqueness of the interpolation polynomial of the smallest degree,\\
\[
= f[x_i, x_i, x_1, \ldots, x_n] = f[x_1, x_2, \ldots, x_i, x_i, \ldots, x_n]
\]


\section*{IX}
Let \( t = \frac{x - \frac{a+b}{2}}{\frac{b-a}{2}} \)
\( x = \frac{b-a}{2}t + \frac{b+a}{2} \)\\
\( \min \max_{x \in [a, b]} \left| a_0 x^n + a_1 x^{n-1} + \cdots + a_n \right| \)\\
\( = \min \max_{t \in [-1, 1]} \left| a_0 \left(\frac{b-a}{2}\right)^n t^n + a_1' t^{n-1} + \cdots + a_n' \right| \)\\
\( = \min \max_{t \in [-1, 1]} \left| a_0 \right| \left(\frac{b-a}{2}\right)^n \left| t^n + \frac{a_1'}{a_0 \left(\frac{b-a}{2}\right)^n} t^{n-1} + \cdots + \frac{a_n'}{a_0 \left(\frac{b-a}{2}\right)^n} \right| \)\\
By Theorem 2.47:\\
Denote by \( \tilde{P}_{n} \) the class of all polynomials of degree \( n \in \mathbb{N}^{+} \) with leading coefficient 1. Then\\
\[
\forall p \in \tilde{P}_{n}, \quad \max_{x \in [-1,1]} \left| \frac{T_{n}(x)}{2^{n-1}} \right| \leq \max_{x \in [-1,1]} \left| p(x) \right|.
\]

\textbf{Brief Restatement of Proof}. By Theorem 2.45, \( T_{n}(x) \) assumes its extrema \( n+1 \) times at the points \( x_{k}^{\prime} \). Suppose the above conclusion does not hold. Then\\
\[
\exists p \in \tilde{P}_{n} \quad \text{s.t.} \quad \max_{x \in [-1,1]} \left| p(x) \right| < \frac{1}{2^{n-1}}.
\]
Then the polynomial \( Q(x) = \frac{1}{2^{n-1}} T_n(x) - p(x) \) satisfies:
\[
Q(x_{k}^{\prime}) = \frac{(-1)^{k}}{2^{n-1}} - p(x_{k}^{\prime}), \quad k = 0, 1, \ldots, n.
\]
\( Q(x) \) changes sign at \( n+1 \) points, having at least \( n \) zeros. However, the \( n \)-th term coefficients of \( \frac{1}{2^{n-1}} T_n(x) \) and \( p(x) \) are the same, the degree of \( Q(x) \) is at most \( n-1 \). Therefore, \( Q(x) \equiv 0 \) and \( p(x) = \frac{1}{2^{n-1}} T_{n}(x) \), which implies \( \max |p(x)| = \frac{1}{2^{n-1}} \), a contradiction.
Thus
\[
\min \max_{t \in [-1, 1]} \left| t^n + \frac{a_1'}{a_0 \left(\frac{b-a}{2}\right)^n} t^{n-1} + \cdots + \frac{a_n'}{a_0 \left(\frac{b-a}{2}\right)^n} \right| = \frac{1}{2^{n-1}}
\]
\[
\min \max_{x \in [a, b]} \left| a_0 x^n + a_1 x^{n-1} + \cdots + a_n \right| = \left| a_0 \right| \left(\frac{b-a}{2}\right)^n \frac{1}{2^{n-1}}
\]
When \( a_0 x^n + a_1 x^{n-1} + \cdots + a_n = a_0 \left(\frac{b-a}{2}\right)^n  T\left(\frac{x - \frac{a+b}{2}}{\frac{b-a}{2}}\right) \) the equality is achieved.

\section*{X}
Prove \( \forall p \in P_n^a \quad \left\| \hat{p}_n \right\|_{\infty} \leq \| p \|_{\infty} \)\\
That is to prove \( \forall p \in P_n^a \) \( \max_{x \in [-1,1]} \left| T_n(x) \right| \leq \max_{x \in [-1,1]} \left| T
(a) \right| \left| p(x) \right| \)
\( \forall p \in P_n^a \)\\
Since \( T(n) \in P_n \), and \( T(n) \) already has \( n \) zeros on \([-1,1]\), \( a > 1 \), thus \( T(a) \neq 0 \)\\
We knew $\max_{x \in [-1,1]} \left| T_n(x) \right|=1$. If the conclusion does not hold\\
\( \exists p \in P_n^a \) such that \( \max_{x \in [-1,1]} \left| p(x) \right| < \frac{1}{T(a)} \)\\
Take this \( p \), let:\\
\( Q(x) = T_n(x) - T(a) p(x) \) \( Q(a) = 0 \).\\
By Theorem 2.45, \( T_{n}(x) \) assumes its extrema \( n+1 \) times at the points \( x_{k}^{\prime} \), the function values at these points are 1 or -1 alternately\\
Thus \( Q(x) \) changes sign at \( n+1 \) points, having at least \( n \) zeros, considering \( Q(a) = 0 \), \( a \notin [-1,1] \), thus \( Q(x) \) has at least \( n+1 \) zeros, but \( Q(x) \in P_n \), thus \( Q(x) = 0 \) always holds.\\
\( p(x) = \frac{T_n(x)}{T(a)} \), \( \max_{x \in [-1,1]} \left| p(x) \right| = \frac{1}{T(a)} \), a contradiction\\
Thus the conclusion is valid.\\

\section*{XI}
By definition\\
\( b_{n,k}(t) := \binom{n}{k} t^k (1-t)^{n-k} \)\\
\( b_{n,k+1}(t) := \binom{n}{k+1} t^{k+1} (1-t)^{n-k-1} \)\\
Substituting, we get\\
\( \frac{n-k}{n} b_{n,k}(t) + \frac{k+1}{n} b_{n,k+1}(t) = \binom{n}{k} \frac{n-k}{n} t^k (1-t)^{n-k} + \frac{k+1}{n} \binom{n}{k+1} t^{k+1} (1-t)^{n-k-1} \)\\
\( = t^k (1-t)^{n-k-1} \left((1-t) \frac{n-k}{n} \binom{n}{k} + t \frac{k+1}{n} \binom{n}{k+1}\right) \)\\
Also:\\
\( (1-t) \frac{n-k}{n} \binom{n}{k} + t \frac{k+1}{n} \binom{n}{k+1} = (1-t+t) \frac{(n-1)!}{k!(n-1-k)!} = \binom{n-1}{k} \)\\
In summary:\\
\( \frac{n-k}{n} b_{n,k}(t) + \frac{k+1}{n} b_{n,k+1}(t) = \binom{n-1}{k} t^k (1-t)^{n-k-1} = b_{n-1,k}(t) \)\\

\section*{XII}
It is required to prove:\\
\( \forall k = 0, 1, \ldots, n, \quad \int_{0}^{1} b_{n,k}(t) \, dt = \frac{1}{n+1} \).\\
\( b_{n,k}(t) := \binom{n}{k} t^k (1-t)^{n-k} \)\\
\( \int_{0}^{1} b_{n,k}(t) \, dt = \binom{n}{k} \int_{0}^{1} t^k (1-t)^{n-k} \, dt \)\\
If \( k = n \)\\
The original formula = \( \int_{0}^{1} t^n \, dt = \frac{1}{n+1} \) holds\\
If \( k = 0 \)\\
The original formula = \( \int_{0}^{1} (1-t)^n \, dt = \frac{1}{n+1} \) also holds\\
When \( k \geq 1 \) and \( k \leq n-1 \)\\
\( \int_{0}^{1} t^k (1-t)^{n-k} \, dt = \frac{1}{k+1} t^{k+1} (1-t)^{n-k} \bigg|_{0}^1 - (-1) \int_{0}^{1} \frac{n-k}{k+1} t^{k+1} (1-t)^{n-k-1} \, dt \)\\
\( = \frac{n-k}{k+1} \int_{0}^{1} t^{k+1} (1-t)^{n-k-1} \, dt \)\\
Repeat until the exponent of \( (1-t) \) is 1\\
\( = \frac{(n-k)(n-k-1) \ldots 2}{(k+1)(k+2) \ldots (n-1)} \int_{0}^{1} t^{n-1} (1-t) \, dt \)\\
\( = \frac{(n-k)(n-k-1) \ldots 2}{(k+1)(k+2) \ldots (n-1)} \left( \frac{1}{n} - \frac{1}{n+1} \right) \)\\
\( = \frac{1}{\binom{n}{k}} \frac{1}{n+1} \)\\
Thus \( \int_{0}^{1} b_{n,k}(t) \, dt = \binom{n}{k} \int_{0}^{1} t^k (1-t)^{n-k} \, dt = \frac{1}{n+1} \)\\
The conclusion is valid.\\




\section*{ \center{\normalsize {Acknowledgement}} }
Use GPT-4 for quick template transformation, and use Kimi AI to correct English grammar.


\end{document}


