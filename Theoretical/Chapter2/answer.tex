\documentclass[a4paper]{article}
%\usepackage[affil-it]{authblk}
\usepackage[backend=bibtex,style=numeric]{biblatex}
\usepackage{amsmath}
\usepackage{geometry}
\usepackage{caption}
\usepackage{amssymb}
\geometry{margin=1.5cm, vmargin={0pt,1cm}}
\setlength{\topmargin}{-1cm}
\setlength{\paperheight}{29.7cm}
\setlength{\textheight}{25.3cm}

\begin{document}
% ==================================================
\title{Numerical Analysis Homework 1}

\author{Kaicheng Luo 3220103383
  \thanks{Electronic address: \texttt{3220103383@zju.edu.com}}}
%\affil{(Information and Computational Science 2201), Zhejiang University}

\date{Due time: \today}

\maketitle

\begin{abstract}
    Solutions to various numerical analysis problems.
\end{abstract}

% ============================================
\section*{I}
For \( f \in \mathcal{C}^{2}[x_{0}, x_{1}] \) and \( x \in (x_{0}, x_{1}) \), linear interpolation of \( f \) at \( x_{0} \) and \( x_{1} \) yields
\[
f(x) - p_1(f; x) = \frac{f^{\prime\prime}(\xi(x))}{2}(x - x_0)(x - x_1).
\]

Consider the case \( f(x) = \frac{1}{x} \), \( x_{0} = 1 \), \( x_{1} = 2 \).

\begin{enumerate}
    \item Determine \( \xi(x) \) explicitly.
    \item Extend the domain of \( \xi \) continuously from \( (x_{0}, x_{1}) \) to \( [x_{0}, x_{1}] \). Find \( \max \xi(x) \), \( \min \xi(x) \), and \( \max f^{\prime\prime}(\xi(x)) \).
\end{enumerate}

\section*{II}
Let \( P_{m}^{+} \) be the set of all polynomials of degree \( \leq m \) that are non-negative on the real line,
\[
P_{m}^{+} = \left\{ p : p \in P_{m}, \forall x \in \mathbb{R}, p(x) \geq 0 \right\}.
\]

Find \( p \in P_{2n}^{+} \) such that \( p(x_{i}) = f_{i} \) for \( i = 0, 1, \ldots, n \) where \( f_{i} \geq 0 \) and \( x_{i} \) are distinct points on \( \mathbb{R} \).

\section*{III}
Consider \( f(x) = e^{x} \).

\begin{enumerate}
    \item Prove by induction that
    \[
    \forall t \in \mathbb{R}, \quad f[t, t+1, \ldots, t+n] = \frac{(e-1)^{n}}{n!} e^{t}.
    \]
    \item From Corollary 2.22 we know
    \[
    \exists \xi \in (0, n) \text{ s.t. } f[0,1,\ldots, n] = \frac{1}{n!} f^{(n)}(\xi).
    \]
    Determine \( \xi \) from the above two equations. Is \( \xi \) located to the left or to the right of the midpoint \( n/2 \)?
\end{enumerate}



\section*{IV}
Consider $f(0)=5,\, f(1)=3,\, f(3)=5,\, f(4)=12$.

\begin{itemize}
    \item Use the Newton formula to obtain $p_{3}(f;x)$;
    \item The data suggest that $f$ has a minimum in $x\in(1,3)$. Find an approximate value for the location $x_{\text{min}}$ of the minimum.
\end{itemize}

\section*{V}
Consider $f(x)=x^{7}$.

\begin{itemize}
    \item Compute $f[0,1,1,1,2,2]$.
    \item We know that this divided difference is expressible in terms of the 5th derivative of $f$ evaluated at some $\xi\in(0,2)$. Determine $\xi$.
\end{itemize}

\section*{VI}
$f$ is a function on $[0,3]$ for which one knows that

\[
f(0)=1,\quad f(1)=2,\quad f^{\prime}(1)=-1,\quad f(3)=f^{\prime}(3)=0.
\]

\begin{itemize}
    \item Estimate $f(2)$ using Hermite interpolation.
    \item Estimate the maximum possible error of the above answer if one knows, in addition, that $f\in\mathcal{C}^{5}[0,3]$ and $|f^{(5)}(x)|\leq M$ on $[0,3]$. Express the answer in terms of $M$.
\end{itemize}



\section*{VII}
Given
\begin{subequations}
\begin{align*}
\Delta f(x) &= f(x+h) - f(x), \\
\Delta^{k+1} f(x) &= \Delta \Delta^k f(x) = \Delta^k f(x+h) - \Delta^k f(x)
\end{align*}
\end{subequations}
and backward difference by
\begin{subequations}
\begin{align*}
\nabla f(x) &= f(x) - f(x-h), \\
\nabla^{k+1} f(x) &= \nabla \nabla^k f(x) = \nabla^k f(x) - \nabla^k f(x-h).
\end{align*}
\end{subequations}
For \( x_j = x + jh \), prove
\begin{subequations}
\begin{align*}
\Delta^k f(x) &= k! h^k f\left[x_0, x_1, \ldots, x_k\right], \\
\nabla^k f(x) &= k! h^k f\left[x_0, x_{-1}, \ldots, x_{-k}\right].
\end{align*}
\end{subequations}

\section*{VIII}
Assume \( f \) is differentiable at \( x_0 \). Prove
\[
\frac{\partial}{\partial x_0} f[x_0, x_1, \ldots, x_n] = f[x_0, x_0, x_1, \ldots, x_n].
\]
What about the partial derivative with respect to one of the other variables?

\section*{IX}
A min-max problem. For \( n \in \mathbb{N}^+ \) and a fixed \( a_0 \neq 0 \), determine
\[
\min \max_{x \in [a, b]} \left| a_0 x^n + a_1 x^{n-1} + \cdots + a_n \right|
\]
where the minimum is taken over all \( a_i \in \mathbb{R}, i = 1, 2, \ldots, n \).

\section*{X}
Imitate the proof of Chebyshev Theorem.

Express the Chebyshev polynomial of degree \( n \in \mathbb{N} \) as a polynomial \( T_n \) and change its domain from \([-1,1]\) to \(\mathbb{R}\). For a fixed \( a > 1 \), define \( P_n^a := \{ p \in P_n : p(a) = 1 \} \) and a polynomial \( \hat{p}_n(x) \in P_n^a \),
\[
\hat{p}_n(x) := \frac{T_n(x)}{T_n(a)}.
\]
For \( \| f \|_{\infty} = \max_{x \in [-1,1]} |f(x)| \), prove \( \forall p \in P_n^a, \quad \left\| \hat{p}_n \right\|_{\infty} \leq \| p \|_{\infty} \).

\section*{XI}
Give a detailed proof of Lemma 2.53.

\section*{XII}
Give a detailed proof of Lemma 2.55.






\section*{ \center{\normalsize {Acknowledgement}} }
Use GPT-4 for quick template transformation, and use Kimi AI to correct English grammar.


\end{document}


