\documentclass[a4paper]{article}
\usepackage[affil-it]{authblk}
\usepackage[backend=bibtex,style=numeric]{biblatex}

\usepackage{geometry}
\geometry{margin=1.5cm, vmargin={0pt,1cm}}
\setlength{\topmargin}{-1cm}
\setlength{\paperheight}{29.7cm}
\setlength{\textheight}{25.3cm}


\begin{document}
% =================================================
\title{Numerical Analysis homework 1}

\author{罗开诚 3220103383
  \thanks{Electronic address: \texttt{3220103383@zju.edu.com}}}
\affil{(信息与计算科学2201), Zhejiang University }


\date{Due time: \today}

\maketitle

\begin{abstract}
    chapter
\end{abstract}



% ============================================

\section*{I.Consider the bisection method starting with the initial interval [1.5,3.5]. In the following questions “the
 interval” refers to the bisection interval whose width changes across di↵erent loops.\\
 • What is the width of the interval at the nth step?\\
 • What is the supremum of the distance between\\
 the root r and the midpoint of the interval?}

\subsection*{I-a}
$\frac{1}{2^{n-1}}$
\subsection*{I-b}
1

\section*{II. In using the bisection algorithm with its initial interval as $[a_0, b_0]$ with $a_0 > 0$, we want to determine the root with its relative error no greater than $\varepsilon$.
 Prove that this goal of accuracy is guaranteed by the following choice of the number of steps,
\[
n \geq \frac{\log(b_0 - a_0) - \log \varepsilon - \log a_0}{\log 2} - 1.
\]
}

define root as t.\\
  error:  $\frac{b_0-a_0}{2^{n-1}}$\\
  the supremum of relative error after iterate n times  $error=\frac{b_0-a_0}{2^{n-1}*r}$\\
  as $a_0,b_0>0$\\
      $error \in [0,\frac{b_0-a_0}{2^{n-1}*a_0}]$\\
  算了,还是用中文吧.\\
  所以可以得到一个上界$\frac{b_0-a_0}{2^{n-1}*a_0}<\epsilon$\\
  得    $n \geq \frac{log(b_0-a_0)-log(\epsilon)-log(a_0)}{log 2}-1$

  \section*{III}
  \begin{flushleft}
    \begin{table}[htbp]
      \centering
      \caption{iterations}
      \begin{tabular}{|c|c|c|c|c|}
      \hline
      iterations & 1 & 2 & 3 & 4 \\ \hline
    \( f'(x) \) & 16 & 11.1719 & 10.2129 & 10.1686 \\ \hline
    \( x_{n+1} \) & -0.8125 & -0.770804 & -0.768832 & -0.768828 \\ \hline
      \end{tabular}
    \end{table}
  \end{flushleft}
  
  \section*{IV}
  \begin{flushleft}
    由拉格朗日,
    $x_{n+1}-\alpha=x_n-\alpha-\frac{f(x_n)-f(\alpha)}{f^{'}(x_0)}$\\
    $=(x_n-\alpha)(1-\frac{f^{'}(\xi)}{f^{'}(x_0)})$  $\xi \in [min(x_n,\alpha),max(x_n,\alpha)]$\\
    所以可取s=1,C=$(1-\frac{f^{'}(\xi)}{f^{'}(x_0)})$  $\xi \in [min(x_n,\alpha),max(x_n,\alpha)]$\\
    
  
  
  \end{flushleft}
  
  \section*{V}
  \begin{flushleft}
    是的喵。
    有$x \in (-\frac{\pi}{2},\frac{\pi}{2})$时,若$x_0=0$,则直接恒为0。\\
    若$x_0>0$,由数学归纳法 $若x_n>0,x_{n+1}=tan^{-1}(x_n)>0$,故$x_n>0$恒成立。$x_0<0$时同理
    这里以$x_0>0$为例。
    $|x_n|=|tan(x_{n+1})|>|x_{n+1}|$
    又$x_n>0,{x_n}$单调有界,其必收敛。设其收敛于$\alpha$
    对$x_n=tan(x_{n+1})$求极限,得$\alpha=tan(\alpha),\alpha=0$
    故收敛于0.
  \end{flushleft}
  
  \section*{VI}
  \begin{flushleft}
    用数列极限的方式表示x。
    令$x_0=\frac{1}{p}$
    $x_{n+1}=\frac{1}{p+x_n}$
    取不动点满足的方程,$x_n^2+px_n-1=0$
    求得大于0得解为$\alpha=\frac{-p+\sqrt{p^2+4}}{2}$
    $|x_{n+1}-\alpha|=|\frac{1}{p+x_n}-\alpha|=|\frac{\alpha^2+p\alpha}{p+x_n}-\alpha|=|x_n-\alpha|*\frac{\alpha}{p+x_n}<|x_n-\alpha|*\frac{\alpha}{p}$(易知$x_n>0$)
    $\frac{\alpha}{p}=\frac{-1+\sqrt{1+4/p^2}}{2}<1$
    故收敛于$\alpha$。
  
  \end{flushleft}
  
  \section*{VII}
  \begin{flushleft}
    设根为 t.\\
    误差为  $\frac{b_0-a_0}{2^{n-1}}$\\
    相对误差的一个上界为  $error=\frac{b_0-a_0}{2^{n-1}*r}$\\
    $\frac{b_0-a_0}{2^{n-1}*t}<\epsilon$\\
    得    $n \leq \frac{log(b_0-a_0)-log(\epsilon)-log(t)}{log 2}-1$\\
    \textbf{是否是一个好的衡量标准:}\\
    不是一个好的衡量标准,若跟在0的极小邻域内,即使绝对误差很小,相对误差依然很大。
  \end{flushleft}
  
  \section*{VIII}
  \begin{flushleft}
  \textbf{$\cdot$\\}  
    可以考虑从高阶无穷小入手啊。\\
    计算$\frac{f(x_n+\Delta x)}{(\Delta x)^p}$,令$\Delta x$趋于0的值,若对p=0,1,2,3……k-1均近似为0,p=k时不收敛为0,则为k重零点。\\
  
  \textbf{$\cdot$\\}  
    根为r
    有$f(x_n)=f(r)+\sum_{t=1}^{k-1}\frac{f^{(t)}(r)}{t!}(x_n-r)^t+\frac{f^{(k)}(\xi_1)}{k!}(x_n-r)^k$\\
    $\xi_1 \in [min(r,x_n),max(r,x_n)]$\\
    和$f^{'}(x_n)=f^{'}(r)+\sum_{t=1}^{k-2}\frac{f^{(t+1)}(r)}{t!}(x_n-r)^t+\frac{f^{(k)}(\xi_2)}{(k-1)!}(x_n-r)^{k-1}$
    $\xi_2 \in [min(r,x_n),max(r,x_n)]$\\
    因为$f^{(k)}(r) \neq 0$,所以可取一个极小邻域$\delta$,使得对$x \in [r-\delta,r+\delta], f^{(k)}(x) \neq 0$,且均同号  \hfill(1)
    故$x_{n+1}=x_n-k\frac{f(x_n)}{f^{'}(x_n)}=x_n-k\frac{\frac{f^{(k)}(\xi_1)}{k!}(x_n-r)^k}{\frac{f^{(k)}(\xi_2)}{(k-1)!}(x_n-r)^{k-1}}$\\
    $=x_n-k\frac{x_n-r}{k}\frac{f^{(k)}(\xi_1)}{f^{(k)}(\xi_2)}=x_n-\frac{f^{k}(\xi_1)-f^{k}(\xi_2)+f^{k}(\xi_2)}{f^{k}(\xi_2)}(x_n-r)$\\
    $=r-\frac{f^{k}(\xi_1)-f^{k}(\xi_2)}{f^{k}(\xi_2)}(x_n-r)$\\
    $|x_{n+1}-r|=|\frac{f^{k}(\xi_1)-f^{k}(\xi_2)}{f^{k}(\xi_2)}|*|x_n-r|=|\frac{f^{k+1}(\xi_3)}{f^{k}(\xi_2)}|*|x_n-r|*|\xi_1-\xi_2|$\\
    (由拉格朗日中值定理,$\xi_3 \in [min(\xi_1,\xi_2),max(\xi_1,\xi_2)]$)\\
    $\leq|\frac{f^{k+1}(\xi_3)}{f^{k}(\xi_2)}|*|x_n-r|^2$(因为$\xi_1,\xi_2 \ in [min(r,x_n),max(r,x_n)]$)\\
    令M=$\frac{max_{x\in[r-\delta,r+\delta]f^{k+1}(x)}}{min_{x\in[r-\delta,r+\delta]f^{k}(x)}}$,存在性可由(1)得\\
    再取$\delta_2<\delta$使得$\delta_2*M<1$,令初值在区间$[r-\delta_2,r+\delta_2]$
    此时满足$|x_{n+1}-r|\leq M*|x_n-r|^2$
    且$|x_n-r|\leq M^{2^n-1}|x_0-r|^{2^{n}}$
    
  \end{flushleft}



\section*{ \center{\normalsize {Acknowledgement}} }
Give your acknowledgements here(if any).


\printbibliography

If you are not familiar with \texttt{bibtex}, 
it is acceptable to put a table here for your references.
\end{document}