\documentclass[a4paper]{article}
\usepackage[affil-it]{authblk}
\usepackage[backend=bibtex,style=numeric]{biblatex}
\usepackage{amsmath}
\usepackage{geometry}
\usepackage{caption}
\geometry{margin=1.5cm, vmargin={0pt,1cm}}
\setlength{\topmargin}{-1cm}
\setlength{\paperheight}{29.7cm}
\setlength{\textheight}{25.3cm}

\begin{document}
% ==================================================
\title{Numerical Analysis Programming 1}

\author{Kaicheng Luo 3220103383
  \thanks{Electronic address: \texttt{3220103383@zju.edu.com}}}
\affil{(Information and Computational Science 2201), Zhejiang University}

\date{Due time: \today}

\maketitle

\begin{abstract}
    Solutions to programming assignments.
\end{abstract}

% ============================================
\section*{ Implement the bisection method, Newton’s method, and the secant method in a C++ package. You should
  I.(a) design an abstract base class EquationSolver with
a pure virtual method solve,
(b) write a derived class of EquationSolver for each
method to accomodate its particularities in the con
tract of solving nonlinear equations.}


\section*{ Test your implementation of the bisection method on the
following functions and intervals
  \[
- x^{-1} - \tan x \text{ on } \left[0, \frac{\pi}{2}\right],
\]

\[
\begin{array}{l}
\bullet x^{-1} - \tan x \text{ on } \left[0, \frac{\pi}{2}\right],\\
\bullet x^{-1} - 2^x \text{ on } [0,1],
\end{array}
\]

\[
- 2^{-x} + e^x + 2\cos x - 6 \text{ on } [1,3],
\]

\[
-\left(\frac{x^{3} + 4 x^{2} + 3 x + 5}{2 x^{3} - 9 x^{2} + 18 x - 2}\right) \text{ on } [0,4].
\]}

\ProgramOutput{ProblemB}

\section*{Test your implementation of Newton's method by solving $x = \tan x$ Find the roots near 4.5 and 7.7.}
\ProgramOutput{ProblemC}


\section*{Test your implementation of the secant method by the following functions and initial values:
\begin{itemize}
    \item $\sin\left(\frac{x}{2}\right) - 1$ with $x_0 = 0, x_1 = \frac{\pi}{2}$,
    \item $e^x - \tan x$ with $x_0 = 1, x_1 = 1.4$,
    \item $x^3 - 12x^2 + 3x + 1$ with $x_0 = 0, x_1 = -0.5$.
\end{itemize}
You should play with other initial values and (if you get different results) think about the reasons.}
\ProgramOutput{ProblemD}

\section*{A trough of length $L$ has a cross section in the shape of a semi-circle with radius $r$. When filled to within a distance $h$ of the top, the water has the volume
\[
V = L\left[0.5\pi r^{2} - r^{2}\arcsin\frac{h}{r} - h\left(r^{2} - h^{2}\right)^{\frac{1}{2}}\right].
\]
Suppose $L = 10 \text{ft}$, $r = 1 \text{ft}$, and $V = 12.4 \text{ft}^3$. Find the depth of water in the trough to within $0.01 \text{ft}$ by each of the three implementations in A.}
\ProgramOutput{ProblemE}

\section*{In the design of all-terrain vehicles, it is necessary to consider the failure of the vehicle when attempting to negotiate two types of obstacles. One type of failure is called hang-up failure and occurs when the vehicle attempts to cross an obstacle that causes the bottom of the vehicle to touch the ground. The other type of failure is called nose-in failure and occurs when the vehicle descends into a ditch and its nose touches the ground.

The above figure shows the components associated with the nose-in failure of a vehicle. The maximum angle $\alpha$ that can be negotiated by a vehicle when $\beta$ is the maximum angle at which hang-up failure does not occur satisfies the equation
\[
A\sin\alpha\cos\alpha + B\sin^2\alpha - C\cos\alpha - E\sin\alpha = 0,
\]
where
\[
\begin{aligned}
A &= l\sin\beta_1, \quad B = l\cos\beta_1, \\
C &= (h + 0.5 D)\sin\beta_1 - 0.5 D\tan\beta_1, \\
E &= (h + 0.5 D)\cos\beta_1 - 0.5 D.
\end{aligned}
\]

(a) Use Newton's method to verify $\alpha \approx 33^{\circ}$ when $l = 89$ in., $h = 49$ in., $D = 55$ in., and $\beta_1 = 11.5^{\circ}$.

(b) Use Newton's method to find $\alpha$ with the initial guess $33^{\circ}$ for the situation when $l, h, \beta_1$ are the same as in part (a) but $D = 30$ in.

(c) Use the secant method (with another initial value as far away as possible from $33^{\circ}$) to find $\alpha$. Show that you get a different result if the initial value is too far away from $33^{\circ}$; discuss the reasons.}
\ProgramOutput{ProblemF}

\section*{ \center{\normalsize {Acknowledgement}} }
Use GPT-4 for quick template transformation, and use Kimi AI to correct English grammar.


\end{document}


